\chapter{FSM Comparison}
\section{FURPS}

\textbf{FSM Framework Design Goals}
\begin{itemize}
\item \textbf{Interoperability}\\
    State machines are typically just an abstraction for describing the logic of a system
    targeted at some problem domain(s) other than FSM construction. We'd like to be able to use libraries
    built for those domains in the implementation of our FSMs, so we want to be sure we can comfortably
    interoperate with other DSELs.

\item \textbf{Declarativeness}\\
    State machine authors should have the experience of describing the structure of
    FSMs rather than implementing their logic. Ideally, building a new state machine should involve little
    more than transcribing its stat transition table into a C++ program. As framework providers, we should be able to
    seamlessly change the implementation of a state machine's logic without affecting the author's
    description.

\item \textbf{Expressiveness}\\
    It should be easy both to represent and to recognize the domain abstraction in a
    program. In our case, an state transitoin table in code should look very much as it does when we design a state
    machine on paper.

\item \textbf{Efficiency}\\
    A simple FSM should ideally compile down to extremely tight code
    that can be optimized into something appropriate even for a tiny embedded system. Perhaps more
    importantly, concerns about the efficiency of our framework should never give programmers an
    excuse for using ad hoc logic where the sound abstraction of a finite state machine might otherwise
    apply.

\item \textbf{Static Type Safety}\\
    It's important to catch as many problems as possible at compile time. A typical
    weakness of many traditional FSM designs is that they do most of their checking at
    runtime. In particular, there should be no need for unsafe downcasts to access the different datatypes
    contained by various events.

\item \textbf{Maintainability}\\
    Simple changes to the state machine design should result in only simple changes to
    its implementation. This may seem like an obvious goal, but it's nontrivial to attainexperts have tried
    and failed to achieve it. For example, when using the State design pattern, a single change
    such as adding a transition can lead to refactoring multiple classes.

\item \textbf{Scalability}\\
    FSMs can grow to be really complex, incorporating
    such features as per-state entry and exit actions, conditional transition guards, default and triggerless
    transitions and even sub-states. If the framework doesn't support these features today, it should be
    reasonably extensible to do so tomorrow.
\end{itemize}

\newpage
\textbf{FURPS} is an acronym representing a model for classifying software quality attributes (functional and non-functional requirements):
\begin{itemize}
\item \textbf{F}unctionality - Feature set, Capabilities, Generality, Security
\item \textbf{U}sability - Human factors, Aesthetics, Consistency, Documentation
\item \textbf{R}eliability - Frequency/severity of failure, Recoverability, Predictability, Accuracy, Mean time to failure
\item \textbf{P}erformance - Speed, Efficiency, Resource consumption, Throughput, Response time
\item \textbf{S}upportability - Testability, Extensibility, Adaptability, Maintainability, Compatibility, Configurability, Serviceability, Installability, Localizability, Portability
\end{itemize}

\subsection{FURPS and FSM Frameworks}

\begin{itemize}
\item \textbf{Functionality}
    \begin{enumerate}[a)]
        \item UML features / extensions
            \begin{itemize}

                \item State machine
                    \begin{itemize}
                        \item Nested FSMs
                        \item Orthogonal regions
                        \item Shallow history
                        \item Deep history
                    \end{itemize}

                \item States
                    \begin{itemize}
                        \item Normal state
                        \item Initial state
                        \item Final state
                        \item Super state
                        \item Pseudo state
                    \end{itemize}

                \item Events
                    \begin{itemize}
                        \item Call event
                        \item Internal event
                    \end{itemize}

                \item Reactions
                    \begin{itemize}
                        \item Guard
                        \item Transition
                        \item Deferral
                        \item Completion/Anonymous transition
                        \item Internal transition
                        \item Self transition
                        \item Conflicting transition (overlapping guards, internal transition in sub fsm and other state)
                    \end{itemize}

                \item Actions
                    \begin{itemize}
                        \item Entry action
                        \item Exit action
                        \item Transition action
                    \end{itemize}

            \end{itemize}

        \item Extra features
            \begin{itemize}
                \item Logging (frameworks supports logging events)
            \end{itemize}
    \end{enumerate}

\item \textbf{Usability}
    \begin{enumerate}[a)]
        \item User interface
            \begin{itemize}
                \item Simple
                \item Intuitive
                \item Not redundant
            \end{itemize}
    \end{enumerate}

\item \textbf{Repeatability}
    \begin{enumerate}[a)]
        \item Exception safety
    \end{enumerate}

\item \textbf{Performance}
    \begin{enumerate}[a)]
        \item Event processing (dispatching) time
        \item Binary size (debug, release)
        \item Runtime memory consumption (valgrind)
        \item Compilation time
    \end{enumerate}

\item \textbf{Supportability}
    \begin{enumerate}[a)]
        \item Testability (dependency injection)
    \end{enumerate}

\item \textbf{Others}
    \begin{enumerate}[a)]
        \item UML compatibility
        \item UML state diagram generation (how easy is to generate state diagram / scripts existence)
    \end{enumerate}

\end{itemize}
\begin{table}[ht]

\newcommand{\yes}{\cellcolor{green}\checkmark}
\newcommand{\simple}{\cellcolor{green}simple}
\newcommand{\some}{\cellcolor{yellow}no superstate}
\newcommand{\no}{\cellcolor{red}-}
\newcommand{\verify}{\cellcolor{yellow}?}
\newcommand{\x}{\cellcolor{white}x}
\newcommand{\notall}{\cellcolor{orange}not all}

\caption{FURPS FSM comparison}
\centering
\begin{tabular}{| l | l | c | c | c | c | c |}
\hline
FURPS &                             & MSM       & StateChart    & QFsm       \\
\hline
\multirow{6}{*}{State machine}
& Nested FSMs                       & \yes      & \yes          & \yes       \\
& Orthogonal regions                & \yes      & \yes          & \yes       \\
& Shallow history                   & \yes      & \yes          & \yes       \\
& Deep history                      & \yes      & \yes          & \yes       \\
\hline
\multirow{5}{*}{States}
& Normal state                      & \yes      & \yes          & \yes       \\
& Initial state                     & \yes      & \yes          & \yes       \\
& Final state                       & \yes      & \yes          & \yes       \\
& Super state                       & \no       & \no           & \no        \\
& Pseudo state                      & \yes      & \no           & \yes       \\
\hline
\multirow{3}{*}{Events}
& Call event                        & \yes      & \yes          & \yes       \\
& Internal event                    & \yes      & \yes          & \yes       \\
\hline
\multirow{7}{*}{Reactions}
& Guard                             & \yes      & \yes          & \yes       \\
& Transition                        & \yes      & \yes          & \yes       \\
& Deferral                          & \yes      & \yes          & \yes       \\
& Completion transition             & \yes      & \yes          & \yes       \\
& Internal transition               & \yes      & \yes          & \yes       \\
& Self transition                   & \yes      & \no           & \yes       \\
& Conflicting transition            & \yes      & \no           & \yes       \\
\hline
\multirow{3}{*}{Actions}
& Entry action                      & \yes      & \yes          & \yes       \\
& Exit action                       & \yes      & \yes          & \yes       \\
& Transition action                 & \yes      & \yes          & \yes       \\
\hline
\multirow{1}{*}{Extra features}
& Logging                           & \no       & \no           & \yes       \\
\hline
\multirow{3}{*}{User interface}
& Simple                            & \yes      & \yes          & \yes       \\
& Intuitive                         & \yes      & \no           & \yes       \\
& Not redundant                     & \yes      & \no           & \yes       \\
\hline
\multirow{1}{*}{Repeatability}
& Exception safety                  & \yes      & \yes          & \yes       \\
\hline
\multirow{2}{*}{Supportability}
& Testability                       & \yes      & \no           & \yes       \\
\multirow{2}{*}{Others}
& UML compatibility                 & \yes      & \yes          & \yes       \\
& UML state diagram generation      & \simple   & \no           & \yes       \\
\hline
\end{tabular}
\end{table}

\begin{table}[ht]

\newcommand{\yes}{\cellcolor{green}\checkmark}
\newcommand{\simple}{\cellcolor{green}simple}
\newcommand{\some}{\cellcolor{yellow}no superstate}
\newcommand{\no}{\cellcolor{red}-}
\newcommand{\verify}{\cellcolor{yellow}?}
\newcommand{\x}{\cellcolor{white}x}
\newcommand{\notall}{\cellcolor{orange}not all}
\caption{FSM Framework Design Goals}
\centering
\begin{tabular}{| l | c | c | c | c |}
\hline
FSM Framework Design Goal         & MSM       & StateChart    & QFsm       \\
\hline
Interoperability                  & \yes      & \yes          & \yes       \\
\hline
Declarativeness                   & \yes      & \notall       & \yes       \\
\hline
Expressiveness                    & \yes      & \yes          & \yes       \\
\hline
Efficiency                        & \yes      & \no           & \yes       \\
\hline
Static Type Safety                & \yes      & \no           & \yes       \\
\hline
Maintainability                   & \yes      & \yes          & \yes       \\
\hline
Scalability                       & \yes      & \yes          & \yes       \\
\hline
\end{tabular}
\end{table}

